\documentclass{article} % (short report)

\begin{document}

\newcommand{\courseinfo}[1]{\\Course: #1}  

\title{Implementation of a BNF to CNF Converter}
\author{Rocco Tescaro} 
\date{March 28, 2024} 
\maketitle  

\courseinfo{B003725 Intelligenza Artificiale (2023/24)} 

\begin{abstract}
Il report descrive l'implementazione dell'algoritmo che verifica la grammatica Backus-Naur Form (BNF) e converte le espressioni BNF valide in Forma Normale Congiuntiva (CNF). Il report dettaglia le funzionalità implementate e le aree potenziali per il miglioramento.
\end{abstract}

\section{Introduction} 

Come esemplificato dalla 

\section{Implementation}  

This section describes the implemented algorithm for BNF verification and conversion to CNF. 

* Describe the steps involved in the algorithm. 
* Mention the chosen data structures and their usage.
* Briefly explain how the code handles invalid BNF syntax.

\section{Improvements} 

This section discusses potential improvements to the current implementation. You can mention:

* Handling more complex BNF grammar features. 
* Optimizing the conversion process for efficiency.
* Implementing error handling and user feedback.

\section{Conclusion}  

This section summarizes the implemented functionalities and highlights the potential areas for improvement.

\end{document}