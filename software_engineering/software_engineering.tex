% Progetto di ingegneria del software - Relazione (pdf)

\documentclass[twoside,openright,titlepage,fleqn,headinclude,12pt,a4paper,BCOR=5mm,footinclude]{scrbook}

\usepackage[nottoc]{tocbibind}
\usepackage[utf8]{inputenc} 
\usepackage[T1]{fontenc} 
\usepackage[english]{babel}
\usepackage{csquotes}
\usepackage[fleqn]{amsmath}  
\usepackage{ellipsis}
\usepackage{listings}
\usepackage{subfig}
\usepackage{caption}
\usepackage{appendix}
\usepackage{siunitx}
\usepackage[pdftex]{graphicx}
\usepackage[eulerchapternumbers,linedheaders,subfig,beramono,eulermath,parts,dottedtoc]{classicthesis}
\usepackage{geometry}

\newlength{\abcd} 
\setlength{\extrarowheight}{3pt} 
\captionsetup{format=hang,font=small}

\geometry{
	a4paper,
	ignoremp,
	bindingoffset = 1cm, 
	textwidth     = 13.5cm,
	textheight    = 21.5cm,
	lmargin       = 3.5cm, 
	tmargin       = 4cm    
}

\lstset{
  	frame=tb,
	language=Matlab,
  	aboveskip=3mm,
  	belowskip=3mm,
  	showstringspaces=false,
  	columns=flexible,
  	basicstyle={\small\ttfamily},
  	numbers=none,
  	breaklines=true,
  	breakatwhitespace=true,
  	tabsize=3
}

\begin{document}
\frenchspacing
\raggedbottom
\pagenumbering{roman}
\pagestyle{plain}

\begin{titlepage}
    \begin{center}
        \large
        \hfill        
        \vfill
        \begingroup
            \includegraphics[scale=0.15]{images/logo}\\
            Scuola di Ingegneria\xspace\\
            Corso di Laurea in Ingegneria Informatica\xspace\\
            \vspace{0.5cm}
            \vspace{0.5cm}
            \textbf{Progetto di Ingegneria del Software}\\
            \vspace{0.5cm}
        \endgroup
        \vfill
        \begingroup
        Applicazione per la gestione di campionati Fanta-sports \xspace \\ $\ $\\
	\bigskip
	\bigskip
      \endgroup
      \spacedlowsmallcaps{Rialti Jacopo, Tescaro Rocco \xspace}
      \vfill 
      \vfill
	\bigskip
	\bigskip
      \vfill
      \vfill
      \vfill
      \vfill
      \vfill
      \vfill
      \vfill
      \vfill
      Anno Accademico 2023-2024
	\end{center}        
\end{titlepage}   

\newpage
\thispagestyle{empty}
\hfill
\vfill
\noindent Rialti Jacopo, Tescaro Rocco \xspace: 
\textit{Applicazione per la gestione di campionati Fanta-sports \xspace,} 
Ingegneria del Software \xspace, \textcopyright\ Anno Accademico 2023-2024 \xspace

\pagestyle{scrheadings}
\cleardoublepage
\clearpage
\pagenumbering{arabic}
\tableofcontents
\cleardoublepage

\clearpage
\chapter{Introduzione}
\section{Obiettivi del progetto}
Il progetto corrente nasce con l'obiettivo di realizzare un'applicazione per la gestione di campionati Fanta-sports, in particolare di calcio.
E' ormai diffusissimo il fenomeno dei campionati fanta-sports, dove i giocatori hanno la possibilità di amministrare le proprie squadre favorite e 
sfidarsi con altri giocatori. L'applicazione nasce appunto con l'obiettivo di fornire un servizio/strumento per l'organizzazione di campionati privati e
per estensione anche pubblici.

L'applicazione permette di creare leghe private di fanta-soccer, ovvero un campionato in cui gli utenti ad invito possono creare una squadra virtuale, 
composta da giocatori reali, e sfidarsi, con anche la possibilità di estendere le regole del campionato personalizzandole o possibilmente collegare
un proprio database all'applicazione rendendo possibile la creazione di campionati di altri sport.

Le principali funzionalità dell'applicazione sono: la possibilità di creare un campionato, di invitare altri utenti a partecipare, di avviare un'asta 
e comporre una squadra, di definire modalità di punteggio e regole del campionato, di visualizzare la classifica e i risultati delle partite.

\section{Elementi del progetto}
Il progetto è composto da:
\begin{itemize}
    \item \textbf{Utente giocatore}: è l'utente che partecipa al campionato, partecipare all'asta, comporre e gestire la propria squadra, 
    visualizzare la classifica e i risultati delle partite. Non può modificare creare un campionato o modificare le regole di uno esistente 
    (nemmeno se partecipante dello stesso).
    \item \textbf{Utente amministratore}: è l'utente che crea un campionato, invita altri utenti a partecipare, avvia l'asta, modifica le regole. 
    Scelte le regole, l'utente amministratore non può più modificarle a campionato iniziato. Ha anche il compito di fornire il database dei giocatori 
    (reali) e il database delle valutazioni per il calcolo del punteggio. 
\end{itemize}

\section{Estendibilità del progetto}
Il progetto è stato pensato per essere estendibile, in particolare per quanto riguarda la possibilità di creare campionati di altri sport.
Per fare ciò è necessario fornire un database di giocatori e un database di valutazioni per il calcolo del punteggio.

Un'altra possibilità di estensione è la possibilità di avviare leghe pubbliche, ovvero leghe il cui accesso è disponibile a tutti gli utenti 
dell'applicazione, senza quindi necessità di invito da parte dell'amministratore. A questa estensione è legata anche la possibilità di visualizzare una
classifica globale, ovvero una classifica che tiene conto di tutti i campionati pubblici attivi.

In ultima istanza è possibile estendere l'applicazione affinchè siano gestite altre regole o tipologie di campionati, idealmente non vincolare dalla regola
dell'immutabilità delle regole a campionato iniziato.

\chapter{Analisi dei requisiti}
\section{Use Case Diagram}

\end{document}

